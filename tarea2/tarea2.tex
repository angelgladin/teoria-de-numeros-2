%%%
 %
 % Copyright (C) 2020 Ángel Iván Gladín García
 %
 % This program is free software: you can redistribute it and/or modify
 % it under the terms of the GNU General Public License as published by
 % the Free Software Foundation, either version 3 of the License, or
 % (at your option) any later version.
 %
 % This program is distributed in the hope that it will be useful,
 % but WITHOUT ANY WARRANTY; without even the implied warranty of
 % MERCHANTABILITY or FITNESS FOR A PARTICULAR PURPOSE.  See the
 % GNU General Public License for more details.
 %
 % You should have received a copy of the GNU General Public License
 % along with this program.  If not, see <http://www.gnu.org/licenses/>.
%%%

%%%%%%%%%%%%%%%%%%%%%%%%%%%%%%%%%%%%%%%%%%%%%%%%%%%%%%%%%%%%%%%%%%%%%%%%%%%%%%%%%%%%%%%%%
\documentclass[letterpaper]{article}
\usepackage[margin=.75in]{geometry}
\usepackage[utf8]{inputenc}
\usepackage[spanish]{babel}
\decimalpoint

\usepackage{listings}
\usepackage{color}
\usepackage{graphicx}
\usepackage{enumerate}
\usepackage{enumitem}
\usepackage{float}

\usepackage{longtable}
\usepackage{hyperref}
\usepackage{commath}

\usepackage{bbm}
\usepackage{dsfont}
\usepackage{mathrsfs}
\usepackage{amsmath,amsthm,amssymb}
\usepackage{mathtools}
\usepackage{longtable}

\usepackage{tikz}
\usetikzlibrary{trees}
\usepackage{verbatim}

%%%%%%%%%%%%%%%%%%%%%%%%%%%%%%%%%%%%%%%%%%%%%%%%%%%%%%%%%%%%%%%%%%%%%%%%%%%%%%%%%%%%%%%%%%%%%%%%5

\usepackage{import}

\usepackage[utf8]{inputenc}

\usepackage{listings}
\usepackage{color}

\usepackage{wasysym}

%%%%%%%%%%%%%%%%%%%%%%%%%%%%%%%%%%%%%%%%%%%%%%%%%%%%%%%%%%%%%%%%%%%%%%%%%%%%%%%%%%%%%%%%%


%%%%%%%%%%%%%%%%%%%%%%%%%%%%%%%%%%%%%%%%%%%%%%%%%%%%%%%%%%%%%%%%%%%%%%%%%%%%%%%%%%%%%%%%%
\newcommand{\Z}{\mathbb{Z}}
\newcommand{\N}{\mathbb{N}}
\newcommand{\Q}{\mathbb{Q}}
\newcommand{\R}{\mathbb{R}}
\newcommand{\Pro}{\mathds{P}}
\newcommand{\Oh}{\mathcal{O}} %% Notacion "O"
\newcommand{\lra}{\longrightarrow}
\newcommand{\ra}{\rightarrow}
\newcommand{\ord}{\text{ord}}
\newcommand{\sol}{\textbf{\underline{Solución}: }} %% Solucion
\newcommand{\af}{\textbf{\underline{Afirmación}: }}
\newcommand{\cej}{\textbf{\underline{Contraejemplo}: }}
\newcommand{\floor}[1]{\lfloor #1 \rfloor}

\DeclareMathOperator{\lcm}{lcm}
\DeclareMathOperator{\rad}{rad}


%%%%%%%%%%%%%%%%%%%%%%%%%%%%%%%%%%%%%%%%%%%%%%%%%%%%%%%%%%%%%%%%%%%%%%%%%%%%%%%%%%%%%%%%%

\begin{document}

%%%%%%%%%%%%%%%%%%%%%%%%%%%%%%%%%%%%%%%%%%%%%%%%%%%%%%%%%%%%%%%%%%%%%%%%%%%%%%%%%%%%%%%%%
\title{
    \vspace{-2.2em}
        Universidad Nacional Autónoma de México\\
        Facultad de Ciencias\\
        Teoría de los números II\\
    \vspace{.5cm}
    \large
        \textbf{Tarea 2}
}
\author{
    Ángel Iván Gladín García\\
    No. cuenta: 313112470\\
    \texttt{angelgladin@ciencias.unam.mx}
}
\date{13 de Abril 2019}
\maketitle
%%%%%%%%%%%%%%%%%%%%%%%%%%%%%%%%%%%%%%%%%%%%%%%%%%%%%%%%%%%%%%%%%%%%%%%%%%%%%%%%%%%%%%%%%

%%%%%%%%%%%%%%%%%%%%%%%%%%%%%%%%%%%%%%%%%%%%%%%%%%%%%%%%%%%%%%%%%%%%%%%%%%%%%%%%%%%%%%%%%
\newtheorem{theorem}{Teorema}
\newtheorem{example}{Ejemplo}
\newtheorem{corollary}{Corolario}
\newtheorem{lemma}{Lemma}
\newtheorem{definition}{Definicion}
\newtheorem{prop}{Proposicion}
%%%%%%%%%%%%%%%%%%%%%%%%%%%%%%%%%%%%%%%%%%%%%%%%%%%%%%%%%%%%%%%%%%%%%%%%%%%%%%%%%%%%%%%%%

%%%%%%%%%%%%%%%%%%%%%%%%%%%%%%%%%%%%%%%%%%%%%%%%%%%%%%%%%%%%%%%%%%%%%%%%%%%%%%%%%%%%%%%%%
\begin{enumerate}

%%%%%%%% 1
\item Sean $f$ y $g$ dos funciones aritméticas y multiplicativas, demuestra que $f \ast g$ es multiplicativa.
\begin{proof}
Sean $n, m \in \Z^+$ y $(n,m) = 1$.
\begin{align*}
    (f \ast g) (nm)
        &= f(nm) \ast g(nm)\\
        &= f(n) \ast f(m) \ast g(n) \ast g(m) && \text{Def. de func. aritmética multiplicativa}\\
        &= f(n) \ast g(n) \ast f(m) \ast g(m) && \text{Conmutatividad en $\mathbb{C}$}\\
        &= (f \ast g) (n) \ast (f \ast g) (m)
\end{align*}
Ergo $f \ast g$ es multiplicativa.
\end{proof}

%%%%%%%% 2
\item Sea $f \neq 0$ una función aritmética multiplicativa, demuestra que existe su inversa respecto a la
convolución de Dirichlet y es multiplicativa.

Recordemos que si $f$ y $g$ son son funciones aritméticas su producto de Dirichlet (o convolución de
Dirichlet) es la función aritmética $h$ definida por la ecuación como

\[
    h(n) = \sum_{d \mid b} f(d) g \left(\frac{n}{d}\right)
\]

\begin{proof} \textbf{Existencia de inversa respecto a la convolución de Dirichlet,} tal que

\[
    f \ast f^{-1} = f^{-1} \ast f = I
\]

Más aún, $f^{-1}$ está dada por las fórmulas recursivas

\[
    f^{-1}(1) = \frac{1}{f(1)}, \qquad
    f^{-1}(n) = \frac{-1}{f(1)} \sum_{d \mid n, \  d < n} f \left(\frac{n}{d}\right)f^{-1}(d)
        \quad \text{para } n > 1.
\]

\begin{itemize}
    \item Para $n = 1$ se debe cumplir que $(f \ast f^{-1})(1) = I(1)$.
    
    Lo cual se reduce a $f(1)f^{-1}(1) = 1$. Como $f(1) \neq 0$, solo hay una solución, es decir
    $f^{-1}(1) = 1/f(1)$.

    \item Consideremos $n > 1$,
    \begin{align*}
        I(n)
            &= (f \ast g) (n)\\
        0
            &= \sum_{d \mid n} f \left(\frac{n}{d}\right)g(d)\\
            &= f(1)f^{-1}(n) + \sum_{d \mid n, \ d < n} f \left(\frac{n}{d}\right)f^{-1}(d)\\
        -f(1)f^{-1}(n)
            &= \sum_{d \mid n, \ d < n} f \left(\frac{n}{d}\right)f^{-1}(d)\\
        f^{-1}(n)
            &= \frac{-1}{f(1)} \sum_{d \mid n, \ d < n} f \left(\frac{n}{d}\right)f^{-1}(d)
    \end{align*}
    Como $f(1) \neq 0$, entonces se tiene demostrado su existencia y unicidad de $f^{-1}$.
\end{itemize}

Donde la función aritmética $I$ está dada por por,

\[
    I(n) = \left[{\frac{1}{n}}\right] =
    \begin{cases}
        1 & \text{si } n = 1\\
        0 & \text{si } n > 1,
    \end{cases}
\]
que es llamada la función identidad.

Además se tiene que para toda $f$ función aritmética $f \ast I = I \ast f = f$
\begin{proof}
    \begin{equation*}
        (f \ast I) (n) = \sum_{d \mid n} f(d) I\left(\frac{n}{d}\right)
            = \sum_{d \mid n} f(d) \left[\frac{d}{n}\right] = f(n)
    \end{equation*}
    dado que $[d/n] = 0$ si $d < n$.
\end{proof}
\end{proof}

\begin{proof} \textbf{Es multiplicativa.}

La fórmula para la inversa ahora es,
\[
    f^{-1}(n) = - \sum_{d \mid n, \  d < n} f^{-1}(d) f \left(\frac{nm}{d}\right)
\]
Así si $(n,m) = 1$ entonces\footnote{
    Factorizamos la expresión de abajo, i.e., el -1 porque si $f$ es multiplicativa entonces $f(1)=1$.
},
\[
    f^{-1}(nm) = - \sum_{d \mid n, \  d < n} f^{-1}(d) f \left(\frac{nm}{d}\right)
        = - \sum_{d_1 \mid n,\ d_2 \mid m,\  d_1 d_2 < n} f^{-1}(d_1d_2) f \left(\frac{nm}{d_1d_2}\right)
\]

Supongamos por inducción que $f^{-1}$ es multiplicativa con argumentos $< nm$.

\end{proof}

%%%%%%%% 3
\item Decimos que una función aritmética es completamente multiplicativa si para cuales quiera $m, n \in \Z^+$,
se tiene que $f(mn) = f(m)f(n)$.

Demuestra que si $f$ es una función aritmética multiplicativa, entonces es completamente multiplicativa si y
solo si $f^{-1}=f\mu$. (Donde $f^{-1}$ es la inversa respecto a la convolución de Dirichlet).
\begin{proof}
    Sea\footnote{Donde $\mu$ es la función de Möbius definida como $\mu(1) = 1$.\\ Si $n > 1$ escribimos a
    $n = p_1^{\alpha_1} \cdots p_n^{\alpha_n}$. Entonces
    \begin{align*}
        \mu(n) = 
            \begin{cases}
                (-1)^k & \text{si } a_1 = a_2 = \ldots = a_k = 1\\
                0 & \text{e.o.c}
            \end{cases}
    \end{align*}
    Hay que notar que $\mu(n) = 0$ si y sólo si $n$ es tiene un factor cuadrado $> 1$.
    }
    $g(n) = \mu(n)f(n)$. Si $f$ es completamente multiplicativa tenemos que 
    \[
        (g \ast f)(n) = \sum_{d \mid n} \mu(d)f(d) f\left(\frac{n}{d}\right)
            = f(n) \sum_{d \mid n} \mu(d) = f(n) I(n) = I(n)
    \]
    como $f(1) = 1$ y $I(n) = 0$ para $n > 1$. Por tanto $g = f^{-1}$.

    De manera análoga, asumimos $f^{-1}(n) = \mu(n)f(n)$. Para probar que $f$ es completamente multiplicativa
    es suficiente con probar que que $f(p^{\alpha}) = f(p)^\alpha$ para potencias primas. La ecuación
    $f^{-1}(n) = \mu(n)f(n)$ implica que
    \[
        \sum_{d \mid n} \mu(d) f(d) \left(\frac{n}{d}\right) \quad \forall n > 1
    \]
    De ahí, tomando $n = p^\alpha$ se tiene
    \[
        \mu(1)f(1)f(p)^\alpha + \mu(p)f(p)f(a^{\alpha - 1}) = 0
    \]
    del cual encontramos $f(p^\alpha) = f(p)f(p^{\alpha-1})$. Que implica que $f(p^\alpha) = f(p)^\alpha$, así
    $f$ es completamente multiplicativa.
\end{proof}


%%%%%%%% 4
\item Sea $m \in \Z^+$, demuestra que la función $f(n)=(n,m)$ es multiplicativa.
\begin{proof}
Denotemos a $\gcd(a, b)$ como el \emph{máximo común divisor} y a $\lcm(a, b)$ el \emph{mínimo común múltiplo}
de dos enteros $a$ y $b$. Por demostrar que el máximo común divisor con un argumento fijo (el cual es $m$) es
una función multiplicativa.

Sea $n=ab$ con $a, b \in \Z$ tal que $\gcd(a,b) = 1$ entonces,

\begin{align*}
    \gcd(m, ab)
        &= \gcd(m, \lcm(a, b))
            && \text{Como $\gcd(a,b) = 1$, entonces $\lcm(a,b) = ab$}\\
        &= \lcm(\gcd(m,a), \gcd(m,b))
            && \text{El $\gcd$ y $\lcm$ se distribuyen sobre cada uno}\\
        &= \frac{\gcd(m,a) \gcd(m,b)} {\gcd(\gcd(m,a) \gcd(m,b))}
            && \text{Producto del gcd con el lcm}\\
        &= \frac{\gcd(m,a) \gcd(m,b)} {\gcd(m,\gcd(a, \gcd(m,b)))}
            && \text{Máximo común divisor es asociativo}\\
            &= \frac{\gcd(m,a) \gcd(m,b)} {\gcd(m ,\gcd(\gcd(a,b),m))}
            && \text{Máximo común divisor es asociativo}\\
        &= \frac{\gcd(m,a) \gcd(m,b)} {\gcd(n, \gcd(1,n))}
            && \text{Por ser primos relativos}\\
        &= \frac{\gcd (m,a)(m,b)} {\gcd(m,1)}\\
        &= \frac{\gcd(m,a) \gcd(m,b)} {1}\\
        &= \gcd(m,a) \gcd(m,b)
\end{align*}
Ergo, el máximo común divisor con un argumento fijo es una función multiplicativa, i.e.

$$ f(ab) = \gcd(ab, m) = \gcd(a,m) \cdot \gcd(b,m) = f(a) \cdot f(b). $$
\end{proof}

%%%%%%%% 5
\item Sea $m \in \Z^+$, demuestra que la función
\[ f(n) = \frac{\varphi(mn)}{\varphi(m)} \quad \text{es multiplicativa}\]
\begin{proof}
    \Large{\frownie}
\end{proof}

%%%%%%%% 6
\item Denotamos $\left[.\right]$ como la parte entera de cualquier número real. Sea
$f(n)= \left[{\sqrt{n}}\right] - \left[{\sqrt{n-1}}\right]$, demuestra que $f$ es multiplicativa.
\begin{proof}
La función $f$ es la indicadora de cuadrados, i.e. $f(n) = 1$ si $n$ es un cuadraro y $f(n) = 0$ en otro caso.
Sea $n = ab$ donde $(a,b) = 1$. De este modo la factorización de $a$ y $b$ consiste de conjuntos disjuntos
de primos. Entonces si $a$ y $b$ don ambos cuadrados, el producto también es un cuadrado. Y si uno de
$a$ o $b$ no es un cuadrado entonces si producto no es un cuadrado. Por tanto la función $f$ es multiplicativa.
\end{proof}

%%%%%%%% 7
\item Demuestra que

\[ \sum_{d \mid n} \frac{|\mu(d)|}{\varphi(d)} = \frac{n}{\varphi(n)} \]
\begin{proof}
Sea $n = p_1^{\alpha_1} \cdots p_k^{\alpha_k}$ donde $\alpha_i \geq 1$. Usando el hecho de que el cociente de dos
funciones aritméticas multiplicativas es multiplicativa y al serlo se cumple que la función
$f(p_1^{\alpha_1} \cdots p_k^{\alpha_k}) = f(p_1^{\alpha_1}) \cdots f(p_k^{\alpha_k})$.
Además tenemos que para cada numerador evaluado en la función de Möbius no cero se tiene que $|\mu(d)|=1$.

Entonces con las observaciones previas, se tiene que
\begin{align*}
    \sum_{d \mid n} \frac{|\mu(d)|}{\varphi(d)}
        &= \prod_{i=1}^{k} \sum_{j = 0}^{a_i} \frac{|\mu(p_i^j)|}{\varphi(p_i^j)}
            && \text{Evaluando cada una de las combs. de sus factores}\\
        &= \prod_{i=1}^{k} \left( 1 + \frac{1}{p_i - 1} \right)
            && \text{Como $|\mu(d)|=1$ y observando que,}\\
        \intertext{\centering $$
            \frac{1}{p_1 \cdots p_j} = \frac{1}{p_1 \cdots p_j \prod_{i=1}^j \frac{p_i - 1}{p_i}}
            = \frac{1}{(p_1 - 1) \cdots (p_j - 1)}$$}
        &= \prod_{i = 1}^{k} \frac{p_i}{p_i - 1} = \prod_{i = 1}^{k} \frac{1}{\frac{1}{p_i}(p_i - 1)}
            = \prod_{i = 1}^{k} \frac{1}{1 - \frac{1}{p_i}}
                && \text{Reacomodo}\\
        &= \frac{n}{\varphi(n)}
            && \text{Porque $\varphi(n) = n\prod_{p \mid n} \left( 1 - \frac{1}{p}\right)$ y agrupando}
\end{align*}
\end{proof}

%%%%%%%% 8
\item Demuestra que

\[ \sum_{d \mid n} \tau(d)^3 = \left( \sum_{d \mid n} \tau(d) \right)^2 \]

\begin{proof}
Recordando que la función aritmética multiplicativa $\tau$ es el número enteros positivos divosores de $n$,
la cual por sé multiplicativa cumple que, sea $n = p_1^{\alpha_1} \cdots p_k^{\alpha_k}$ entonces
$\tau(n) = \prod_{i=1}^{k}(a_i +1)$.

Sea $n = p_1^{\alpha_1} \cdots p_k^{\alpha_k}$,
\begin{align*}
    \tau(n)^3
        &= \prod_{i=1}^{k} \tau(p_i^{\alpha_i})^3 = \prod_{i=1}^{k} \sum_{j=1}^{a_i} (j_i + 1)^3
            && \text{Por ser multiplicativa y haciendo aritmética}\\
        &= \prod_{i=1}^{k} \left( \sum_{j=1}^{a_i} (j_i + 1) \right)^2
            && \text{Por $(*)$}\\
        &= \prod_{i=1}^{k} \tau(p_i^{\alpha_i})^2\\
        &= (\tau(n))^2
\end{align*}
\end{proof}

Ahora solo hace falta probar $(*)$, ósea
\[
    \sum_{i=1}^{n}i^3 = \left( \sum_{i=1}^{n}i \right)^2
\]
\begin{proof} \textbf{Por inducción}

    Tenemos que $\sum_{i=1}^{n} i = \frac{n(n+1)}{2}$, entonces
    $\left( \sum_{i=1}^{n} i \right)^2 = \frac{n^2(n+1)^2}{4}$.
    Mostrar por inducción que
    \[
        \sum_{i=1}^{n}i^3 = \frac{n^2(n+1)^2}{4}
    \]
    \emph{Paso inductivo con $n=k+1$,}
    \begin{align*}
        \sum_{i=1}^{k+1}i^3
            &= \sum_{i=1}^{k}i^3 + (k+1)^3\\
            &= \frac{k^2(k+1)^2}{4} + (k+1)^3\\
            &= \frac{k^4 + 2k^3 + k^2}{4} + \frac{4k^3 + 12k^2 + 12k + 4} {4}\\
            &= \frac {k^4 + 6k^3 + 13k^2 + 12k + 4} {4}\\
            &= \frac {(k + 1)^2 (k + 2)^2} {4}
    \end{align*}
\end{proof}
Por tanto $\sum_{d \mid n} \tau(d)^3 = \left( \sum_{d \mid n} \tau(d) \right)^2$.
%%%%%%%% 9
\item Sea $f$ una función aritmética, demuestra que existe $g$ una función aritmética y multiplicativa tal que:

\[ \sum_{k=1}^{n} f((k,n)) = \sum_{d \mid n} f(d)g \left( \frac{n}{d} \right) \]

\begin{proof}
Recordando la función aritmética $N$ la cual $N(n)$ para toda $n$ y teniendo la siguiente identidad,
\[
    \sum_{k=1}^{n} (k,n) \mu((k,n)) = \mu(n) \tag{1}
\]

Tomando convenientemente $g = \varphi$, partiendo a $k$ por su máximo común múltiplo con $n$ da

\begin{align*}
    \sum_{k=1}^{n} f((k,n))
        &= \sum_{d \mid n} \sum_{1 \leq k \leq n,\ (k,n)=d} f((k,n))\\
        &= \sum_{d \mid n} f(d) \sum_{1 \leq k \leq n,\ (k,n)=d} 1\\
        &= \sum_{d \mid n} f(d) \varphi \left( \frac{n}{d} \right)\\
\end{align*}

Ahora como $\varphi = N \ast \mu$ y aplicando la identidad $(1)$, se tiene
\[
    f \ast \varphi = (N \ast \mu) \ast (N \ast \mu) = (N^{-1} \ast N) \ast \mu = I \ast \mu = \mu.
\]
\end{proof}

%%%%%%%% 10
\item Sea $n \in \Z^+$, definimos $\rad(n)$ el radical de $n$ como la función multiplicativa que satisface que
$\rad(p^\alpha) = p$ para todo $p$ número primo y $\alpha \geq 1$. Demuestra que

\[ \rad(n) = \sum_{d \mid n} \varphi(d) |\mu(d)| \]

\begin{proof}
Sea $n = p_1^{\alpha_1} \cdots p_k^{\alpha_k}$ con $\alpha_i \geq 1$.

Desarrollando en lado izquierdo de la ecuación,

\begin{align*}
    \rad(p_1^{\alpha_1} \cdots p_k^{\alpha_k})
        &= \rad(p_1^{\alpha_1}) \cdots \rad(p_k^{\alpha_k})
        && \text{Por ser multiplicativa}\\
        &= \prod_{i=1}^k p_i
\end{align*}

Desarrollando en lado derecho de la ecuación,

\begin{align*}
    \sum_{d \mid n} \varphi(d) |\mu(d)|
        &= \prod_{i=1}^{k} \sum_{d \mid p_i^{\alpha_i}} \varphi(d) |\mu(d)|
            && \text{Por ser multiplicativa}\\
        &= \prod_{i=1}^{k} \sum_{d \mid p_i} \varphi(d) |\mu(d)|
        && \text{Por def. de func. de Möbius}\\
            \intertext{\centering
                Además si se tenia $\alpha_i > 1$ su valor es cero y se cancela, y en otro caso 1.}
        &= \prod_{i=1}^{k} (\varphi(1) |\mu(1)|) + (\varphi(p_i) |\mu(p_i)|)
            = \prod_{i=1}^{k} 1 + (p_i-1)
            && \text{Evaluando}\\
        &= \prod_{i=1}^{k} p_i
\end{align*}

Teniendo ambos lados de la ecuación iguales.
\end{proof}

\end{enumerate}



%%%%%%%%%%%%%%%%%%%%%%%%%%%%%%%%%%%%%%%%%%%%%%%%%%%%%%%%%%%%%%%%%%%%%%%%%%%%%%%%%%%%%%%%%


%%%%%%%%%%%%%%%%%%%%%%%%%%%%%%%%%%%%%%%%%%%%%%%%%%%%%%%%%%%%%%%%%%%%%%%%%%%%%%%%%%%%%%%%%

%%%%%%%%%%%%%%%%%%%%%%%%%%%%%%%%%%%%%%%%%%%%%%%%%%%%%%%%%%%%%%%%%%%%%%%%%%%%%%%%%%%%%%%%%

\end{document}