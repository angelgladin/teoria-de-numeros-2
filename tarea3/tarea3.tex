%%%
 %
 % Copyright (C) 2020 Ángel Iván Gladín García
 %
 % This program is free software: you can redistribute it and/or modify
 % it under the terms of the GNU General Public License as published by
 % the Free Software Foundation, either version 3 of the License, or
 % (at your option) any later version.
 %
 % This program is distributed in the hope that it will be useful,
 % but WITHOUT ANY WARRANTY; without even the implied warranty of
 % MERCHANTABILITY or FITNESS FOR A PARTICULAR PURPOSE.  See the
 % GNU General Public License for more details.
 %
 % You should have received a copy of the GNU General Public License
 % along with this program.  If not, see <http://www.gnu.org/licenses/>.
%%%

%%%%%%%%%%%%%%%%%%%%%%%%%%%%%%%%%%%%%%%%%%%%%%%%%%%%%%%%%%%%%%%%%%%%%%%%%%%%%%%%%%%%%%%%%
\documentclass[letterpaper]{article}
\usepackage[margin=.75in]{geometry}
\usepackage[utf8]{inputenc}
\usepackage[spanish]{babel}
\decimalpoint

\usepackage{mathrsfs}
\usepackage{amsmath,amsthm,amssymb}
\usepackage{wasysym}

\usepackage{cite}

%%%%%%%%%%%%%%%%%%%%%%%%%%%%%%%%%%%%%%%%%%%%%%%%%%%%%%%%%%%%%%%%%%%%%%%%%%%%%%%%%%%%%%%%%


%%%%%%%%%%%%%%%%%%%%%%%%%%%%%%%%%%%%%%%%%%%%%%%%%%%%%%%%%%%%%%%%%%%%%%%%%%%%%%%%%%%%%%%%%
\newcommand{\Z}{\mathbb{Z}}
\newcommand{\R}{\mathbb{R}}
%%%%%%%%%%%%%%%%%%%%%%%%%%%%%%%%%%%%%%%%%%%%%%%%%%%%%%%%%%%%%%%%%%%%%%%%%%%%%%%%%%%%%%%%%

\begin{document}

%%%%%%%%%%%%%%%%%%%%%%%%%%%%%%%%%%%%%%%%%%%%%%%%%%%%%%%%%%%%%%%%%%%%%%%%%%%%%%%%%%%%%%%%%
\title{
    \vspace{-2.2em}
        Universidad Nacional Autónoma de México\\
        Facultad de Ciencias\\
        Teoría de los números II\\
    \vspace{.5cm}
    \large
        \textbf{Tarea 3 \emph{Generalización del Postulado de Bertrand}}
}
\author{
    Ángel Iván Gladín García\\
    No. cuenta: 313112470\\
    \texttt{angelgladin@ciencias.unam.mx}
}
\date{17 de junio de 2020}
\maketitle
%%%%%%%%%%%%%%%%%%%%%%%%%%%%%%%%%%%%%%%%%%%%%%%%%%%%%%%%%%%%%%%%%%%%%%%%%%%%%%%%%%%%%%%%%
\subsection*{Posutulado de Bertrand}
Dice que para cualquier entero $n > 1$ siempre existe al menos un primo tal que,
\[
    n < p < 2n.
\]

\subsection*{¿Qué pasa con los interválos $2x$ a $3x$, o $x$ mayor igual que 2.}

El artículo de M. El Bachraoui\cite{Bachraoui} nos habla de los primos en el intervalo $[2n, 3n]$, específicamente
establece en un teorema que para cualquier enteros positivo $n < 1$ hay un número primo entre $2n$ y $3n$.
Lo que hace es afirmar que hay un producto de primos entre $2n$ y $3n$, los cuales, si hay alguno
que dividan a $\binom{3n}{2n}$. Usando una notación de un libro de Erdös tiene que,
\[
    T_1 = \prod_{p \leq \sqrt{3n}} p^{\beta(p)},\quad
    T_2 = \prod_{\sqrt{3n} < p \leq 2n} p^{\beta(p)},\quad
    T_2 = \prod_{2n+1 \leq p \leq 3n} p^{\beta(p)}
\]
tales ques,
\[
    \binom{3n}{2n} = T_1 T_2 T_3
\]
Donde acota cada $T_i$ y al final muestra la existencia de esos números.

\subsection*{¿Qué se puede decir del intervalo $kx$ a $(k+1)x$ con $k>2$?}
El el artículo\cite{shevelev2012intervals} se buscan esas $k$'s tal que se cumpla ese intervalo y se prueba al menos
para $k = 1,2,3,5,9,14$ y no otros, al menos para $k \leq 50,000,000$. Además se tiene para cada $k = 1,2,3,5,9,14$
se tiene un algoritmo para entror el más pequño $N_k(m)$, tal que, para $n \geq N_k(m)$, el intervalo $(kn, (k+1)n)$
contiene al menos $m$ primos.
%%%%%%%%%%%%%%%%%%%%%%%%%%%%%%%%%%%%%%%%%%%%%%%%%%%%%%%%%%%%%%%%%%%%%%%%%%%%%%%%%%%%%%%%%

\bibliographystyle{unsrt}
\bibliography{ref}



%%%%%%%%%%%%%%%%%%%%%%%%%%%%%%%%%%%%%%%%%%%%%%%%%%%%%%%%%%%%%%%%%%%%%%%%%%%%%%%%%%%%%%%%%


%%%%%%%%%%%%%%%%%%%%%%%%%%%%%%%%%%%%%%%%%%%%%%%%%%%%%%%%%%%%%%%%%%%%%%%%%%%%%%%%%%%%%%%%%

%%%%%%%%%%%%%%%%%%%%%%%%%%%%%%%%%%%%%%%%%%%%%%%%%%%%%%%%%%%%%%%%%%%%%%%%%%%%%%%%%%%%%%%%%

\end{document}