%%%
 %
 % Copyright (C) 2020 Ángel Iván Gladín García
 %
 % This program is free software: you can redistribute it and/or modify
 % it under the terms of the GNU General Public License as published by
 % the Free Software Foundation, either version 3 of the License, or
 % (at your option) any later version.
 %
 % This program is distributed in the hope that it will be useful,
 % but WITHOUT ANY WARRANTY; without even the implied warranty of
 % MERCHANTABILITY or FITNESS FOR A PARTICULAR PURPOSE.  See the
 % GNU General Public License for more details.
 %
 % You should have received a copy of the GNU General Public License
 % along with this program.  If not, see <http://www.gnu.org/licenses/>.
%%%

%%%%%%%%%%%%%%%%%%%%%%%%%%%%%%%%%%%%%%%%%%%%%%%%%%%%%%%%%%%%%%%%%%%%%%%%%%%%%%%%%%%%%%%%%
\documentclass[letterpaper]{article}
\usepackage[margin=.75in]{geometry}
\usepackage[utf8]{inputenc}
\usepackage[spanish]{babel}
\decimalpoint

\usepackage{mathrsfs}
\usepackage{amsmath,amsthm,amssymb}
\usepackage{wasysym}

%%%%%%%%%%%%%%%%%%%%%%%%%%%%%%%%%%%%%%%%%%%%%%%%%%%%%%%%%%%%%%%%%%%%%%%%%%%%%%%%%%%%%%%%%


%%%%%%%%%%%%%%%%%%%%%%%%%%%%%%%%%%%%%%%%%%%%%%%%%%%%%%%%%%%%%%%%%%%%%%%%%%%%%%%%%%%%%%%%%
\newcommand{\Z}{\mathbb{Z}}
\newcommand{\R}{\mathbb{R}}
%%%%%%%%%%%%%%%%%%%%%%%%%%%%%%%%%%%%%%%%%%%%%%%%%%%%%%%%%%%%%%%%%%%%%%%%%%%%%%%%%%%%%%%%%

\begin{document}

%%%%%%%%%%%%%%%%%%%%%%%%%%%%%%%%%%%%%%%%%%%%%%%%%%%%%%%%%%%%%%%%%%%%%%%%%%%%%%%%%%%%%%%%%
\title{
    \vspace{-2.2em}
        Universidad Nacional Autónoma de México\\
        Facultad de Ciencias\\
        Teoría de los números II\\
    \vspace{.5cm}
    \large
        \textbf{Tarea 4 \emph{Armónicos y Promedios}}
}
\author{
    Ángel Iván Gladín García\\
    No. cuenta: 313112470\\
    \texttt{angelgladin@ciencias.unam.mx}
}
\date{27 de Mayo de 2019}
\maketitle
%%%%%%%%%%%%%%%%%%%%%%%%%%%%%%%%%%%%%%%%%%%%%%%%%%%%%%%%%%%%%%%%%%%%%%%%%%%%%%%%%%%%%%%%%


%%%%%%%%%%%%%%%%%%%%%%%%%%%%%%%%%%%%%%%%%%%%%%%%%%%%%%%%%%%%%%%%%%%%%%%%%%%%%%%%%%%%%%%%%
\begin{enumerate}

%%%%%%%% 1
\item Demostrar que para la función $\tau(n)$, que es la función cantidad de divisores, se cumple que
\[
    \tau(n) \ll n^{\frac{1}{3}}
\]

\begin{proof}
Consideremos como primer caso potencias de un número primo fijo, $n = p^t$. Como $\tau(p^t) = t + 1$
es menor que $p^{t/3}$ cuando $t$ crece porque $p^{t/3} = \exp(t \log(p) / 3)$ crece exponencialmente en $t$. Así.
\[
    t + 1 \ll t \quad\text{y}\quad t \ll \exp(t\log(p)/3) \quad\text{y así,}\quad
    t + 1 \ll \exp(t\log(p)/3)
\]

Supongamos que $p > e^3$ está fijo, así $\log(p) > 3$. $t+1$ y $\exp(t \log(p) / 3)$ son iguales a 1 cuando
$t = 0$. Para ver cual crece más rápido, cómparemos sus derivadas cuando $t = 0$,

\begin{align*}
    \frac{d}{dt} (t+1) \Bigr|_{t = 0} = 1,
\end{align*}

\begin{align*}
    \frac{d}{dt} (\exp(t \log(p)/3))\Bigr|_{t = 0}
        &= \left( \frac{\log(p)}{3} \exp(t \log(p) / 3) \right)\Bigr|_{t = 0}\\
        &= \frac{\log(p)}{3} > 1
\end{align*}

Entonces la exponenvias crece más rápido cuando $t = 0$, así
\[
    \tau(p^t) = t + 1 \leq p^{t/3} \quad \text{para toda } t,
\]
y para todos los primos $p \geq 23$. Así, $\tau(n) \leq n^{1/3}$, siempre y cuando $n$ solo por primos $p \geq 23$.


Esto sigue dejando a los primos $p = 2,3,5, \ldots, 19$. Para cada uno de esos primos, determinados que la función
$(t+1)p^{-t/3}$ tiene un máximo en $t = 3/\log(p) ,1$, el valor máximo es una contante $C(p)$. Así

\[
    t + 1 \leq C(p)p^{t/3} \quad\text{para toda }t,\quad \text{para } p=2,3,\ldots,19.
\]

Pongamos $C(p) = 1$ para $p > 19$ y sea $C$ el producto de todas las constante $C(p)$. Ahora para
$n = \prod_i p_i^{t_i}$
\[
    \tau(\prod_i p_i^{t_i}) = \prod_i (t_i + 1) \leq \prod_i C(p_i) p_i^{t_i/3} \leq C \prod_i p_i^{t_i/3} = Cn^{1/3}
\]

\end{proof}

%%%%%%%% 2
\item Demostrar que no puede suceder que
\[
    \tau(n) \ll \ln(n)
\]

\begin{proof}
Consideremos a $n$ de la forma $2^m \cdot 3^m$, entonces $\tau(n) = (m+1)^2$.

Ahora $n = 6^m$, entonces $m = \log(n)/\log(6)$ y así $\tau(n) = (\log(n)/\log(6) +1 )^2$. Cambiemos las varibles con
$x = \log(n)$. Ahora lo que queremos mostrar que para cualquier $C$, la desigualdad
\[
    \left(\frac{x}{\log(6)} + 1\right)^2 \leq Cx
\]
es falsa, expandiendo los términos se tiene,
\[
    \frac{x^2}{\log(6)^2} + \left( \frac{2}{\log(6) - C} \right) x + 1 \leq 0
\]
Y como la parte derecha de la desigualdad tiende a infinito, no se cumple que $\tau(n) \ll \ln(n)$.
\end{proof}

%%%%%%%% 3
\item Demostrar que $\ln(n) < H_n < \ln(n) + 1$
\begin{proof}
En clase se demotró la siguiente desigualdad
\begin{equation}
    H_n - 1 < \log(n) < H_{n-1} \text{ para toda } n \geq 1
\end{equation}

La cual es usada para el siguiente teorema
\begin{equation}
    H_n = \log(n) + O(1) \text{ para toda } n \geq 1
\end{equation}

Dando como consecuencia lo siguiente,
\begin{equation}
    0 < H_n - \log(n) < 1
    \label{eq:3.1}
\end{equation}

Pero si tomamos \eqref{eq:3.1} sumando $\log(n)$ en cada término de la desigualdad, se sigue,
\begin{align*}
    0 < H_n - \log(n) < 1\\
    \log(n) < H_n < \log(n) + 1
\end{align*}

\end{proof}

%%%%%%%% 4
\item Demostrar que $n! \ll n \left( \frac{n}{e} \right)^n$
\begin{proof}
Citando un lema visto en clase que establece,
\begin{equation}
    \log(n!) = n\log(n) - n + O(\log(n))
\end{equation}

Teniendo como consecuencia que,
\begin{equation}
    \log(n!) - (n \log(n) - n) \ll \log(n) 
\end{equation}

Más aún, por $O$ sabemos que para alguna $C$ se tiene que
\begin{equation}
    \label{eq:4.1}
    |\log(n!) - (n \log(n) - n)| \leq C\log(n) 
\end{equation}

Pero haciendo algunas observaciones en $n \log(n) - n$,
\begin{align*}
    n \log(n) - n
        &= \log(n^n) - n && \text{Potencia en logarítmo } \log_b(x^p) = p\log_b x\\
        &= \log(n^n) - \log(e^n)\\
        &= \log((n/e)^n) && \text{Cociente en logarítmo } \log_b \frac{x}{y} = \log_b x - \log_b y
\end{align*}

Sustituyedo $n \log(n) - n$ en \eqref{eq:4.1} se tiene,
\begin{equation}
    |\log(n!) - \log((n/e)^n)| \leq C\log(n)
\end{equation}

Lo cual por definición de valor absoluto se sigue que,
\begin{equation}
    -C \log(n) \leq \log(n!) - \log((n/e)^n) \leq C\log(n)
\end{equation}

Tomando $\log(n!) - \log((n/e)^n) \leq C \log(n)$ tal que,
\begin{equation}
    e^{\log(n!) - \log((n/e)^n)} \leq e^{\log{n^C}} \quad\iff\quad
    \frac{n!}{(n/e)^n} \leq n^C
\end{equation}

Teniendo así,
\begin{equation}
    n! \leq n^C (n/e)^n
\end{equation}

Y tomando $C=1$ implica que $n! \ll n \left( \frac{n}{e} \right)^n$.

\end{proof}

%%%%%%%% 5
\item Demostrar sin usar puntos en una retícula, es decir, de manera totalmente algebraica que:
\[
    \sum_{k=1}^{n} \tau(k) = n\log(n) + O(n)
\]

\begin{proof}
Sabemos que
\[
    \sum_{k=1}^{n} \tau(k) = \sum_{k = 1}^n \sum_{d \mid k} 1 = \sum_{\substack{c,d \\ cd \leq n}} 1.
\]
Tomando así todos los divisores de todos los enteros menores o iguales a $n$ que son exactactamente todas las
parejas de enteros $(c,d)$ con $cd \leq n$.

Pero si $cd \leq n$, entonces $d \leq n$ y $c \leq n/d$, así,
\[
    \sum_{k=1}^{n} \tau(k) = \sum_{d \leq n} \sum_{c \leq n/d} 1
\]
Tenemos que $\sum_{c \leq n/d} 1$ cuenta cuandos enteros $c$ son menores o iguales a $n/d$. Esto es $[n/d]$ la
parte entera del número racional $n/d$. Y cuando redondeamos un número cambia por un error menor que 1,
así $[n/d] = n/d + O(1)$. Teniendo así,
\begin{align*}
    \sum_{k=1}^{n} \tau(k)
        &= \sum_{d \leq n} [n/d] = \sum_{d \leq n} \{ n/d + O(1) \}\\
        &= \sum_{d \leq n} n/d + \sum_{d \leq n} O(1) = nH_n + O(n)
\end{align*}
Donde $nH_n$ viene de la definición de números harmónicos. Y $O(n)$ porque el error es a lo más 1 y corre de
$d \leq n$ con $n$ diferentes errores. Como tenemos un error a los más $n$, entonces,
\[
    \sum_{k=1}^{n} = n \{ \log(n)+O(1) \} + O(n) = n\log(n) + O(n)
\]
\end{proof}

\end{enumerate}



%%%%%%%%%%%%%%%%%%%%%%%%%%%%%%%%%%%%%%%%%%%%%%%%%%%%%%%%%%%%%%%%%%%%%%%%%%%%%%%%%%%%%%%%%


%%%%%%%%%%%%%%%%%%%%%%%%%%%%%%%%%%%%%%%%%%%%%%%%%%%%%%%%%%%%%%%%%%%%%%%%%%%%%%%%%%%%%%%%%

%%%%%%%%%%%%%%%%%%%%%%%%%%%%%%%%%%%%%%%%%%%%%%%%%%%%%%%%%%%%%%%%%%%%%%%%%%%%%%%%%%%%%%%%%

\end{document}